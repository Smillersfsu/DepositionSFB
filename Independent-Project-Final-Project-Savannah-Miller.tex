% Options for packages loaded elsewhere
\PassOptionsToPackage{unicode}{hyperref}
\PassOptionsToPackage{hyphens}{url}
%
\documentclass[
]{article}
\usepackage{amsmath,amssymb}
\usepackage{iftex}
\ifPDFTeX
  \usepackage[T1]{fontenc}
  \usepackage[utf8]{inputenc}
  \usepackage{textcomp} % provide euro and other symbols
\else % if luatex or xetex
  \usepackage{unicode-math} % this also loads fontspec
  \defaultfontfeatures{Scale=MatchLowercase}
  \defaultfontfeatures[\rmfamily]{Ligatures=TeX,Scale=1}
\fi
\usepackage{lmodern}
\ifPDFTeX\else
  % xetex/luatex font selection
\fi
% Use upquote if available, for straight quotes in verbatim environments
\IfFileExists{upquote.sty}{\usepackage{upquote}}{}
\IfFileExists{microtype.sty}{% use microtype if available
  \usepackage[]{microtype}
  \UseMicrotypeSet[protrusion]{basicmath} % disable protrusion for tt fonts
}{}
\makeatletter
\@ifundefined{KOMAClassName}{% if non-KOMA class
  \IfFileExists{parskip.sty}{%
    \usepackage{parskip}
  }{% else
    \setlength{\parindent}{0pt}
    \setlength{\parskip}{6pt plus 2pt minus 1pt}}
}{% if KOMA class
  \KOMAoptions{parskip=half}}
\makeatother
\usepackage{xcolor}
\usepackage[margin=1in]{geometry}
\usepackage{color}
\usepackage{fancyvrb}
\newcommand{\VerbBar}{|}
\newcommand{\VERB}{\Verb[commandchars=\\\{\}]}
\DefineVerbatimEnvironment{Highlighting}{Verbatim}{commandchars=\\\{\}}
% Add ',fontsize=\small' for more characters per line
\usepackage{framed}
\definecolor{shadecolor}{RGB}{248,248,248}
\newenvironment{Shaded}{\begin{snugshade}}{\end{snugshade}}
\newcommand{\AlertTok}[1]{\textcolor[rgb]{0.94,0.16,0.16}{#1}}
\newcommand{\AnnotationTok}[1]{\textcolor[rgb]{0.56,0.35,0.01}{\textbf{\textit{#1}}}}
\newcommand{\AttributeTok}[1]{\textcolor[rgb]{0.13,0.29,0.53}{#1}}
\newcommand{\BaseNTok}[1]{\textcolor[rgb]{0.00,0.00,0.81}{#1}}
\newcommand{\BuiltInTok}[1]{#1}
\newcommand{\CharTok}[1]{\textcolor[rgb]{0.31,0.60,0.02}{#1}}
\newcommand{\CommentTok}[1]{\textcolor[rgb]{0.56,0.35,0.01}{\textit{#1}}}
\newcommand{\CommentVarTok}[1]{\textcolor[rgb]{0.56,0.35,0.01}{\textbf{\textit{#1}}}}
\newcommand{\ConstantTok}[1]{\textcolor[rgb]{0.56,0.35,0.01}{#1}}
\newcommand{\ControlFlowTok}[1]{\textcolor[rgb]{0.13,0.29,0.53}{\textbf{#1}}}
\newcommand{\DataTypeTok}[1]{\textcolor[rgb]{0.13,0.29,0.53}{#1}}
\newcommand{\DecValTok}[1]{\textcolor[rgb]{0.00,0.00,0.81}{#1}}
\newcommand{\DocumentationTok}[1]{\textcolor[rgb]{0.56,0.35,0.01}{\textbf{\textit{#1}}}}
\newcommand{\ErrorTok}[1]{\textcolor[rgb]{0.64,0.00,0.00}{\textbf{#1}}}
\newcommand{\ExtensionTok}[1]{#1}
\newcommand{\FloatTok}[1]{\textcolor[rgb]{0.00,0.00,0.81}{#1}}
\newcommand{\FunctionTok}[1]{\textcolor[rgb]{0.13,0.29,0.53}{\textbf{#1}}}
\newcommand{\ImportTok}[1]{#1}
\newcommand{\InformationTok}[1]{\textcolor[rgb]{0.56,0.35,0.01}{\textbf{\textit{#1}}}}
\newcommand{\KeywordTok}[1]{\textcolor[rgb]{0.13,0.29,0.53}{\textbf{#1}}}
\newcommand{\NormalTok}[1]{#1}
\newcommand{\OperatorTok}[1]{\textcolor[rgb]{0.81,0.36,0.00}{\textbf{#1}}}
\newcommand{\OtherTok}[1]{\textcolor[rgb]{0.56,0.35,0.01}{#1}}
\newcommand{\PreprocessorTok}[1]{\textcolor[rgb]{0.56,0.35,0.01}{\textit{#1}}}
\newcommand{\RegionMarkerTok}[1]{#1}
\newcommand{\SpecialCharTok}[1]{\textcolor[rgb]{0.81,0.36,0.00}{\textbf{#1}}}
\newcommand{\SpecialStringTok}[1]{\textcolor[rgb]{0.31,0.60,0.02}{#1}}
\newcommand{\StringTok}[1]{\textcolor[rgb]{0.31,0.60,0.02}{#1}}
\newcommand{\VariableTok}[1]{\textcolor[rgb]{0.00,0.00,0.00}{#1}}
\newcommand{\VerbatimStringTok}[1]{\textcolor[rgb]{0.31,0.60,0.02}{#1}}
\newcommand{\WarningTok}[1]{\textcolor[rgb]{0.56,0.35,0.01}{\textbf{\textit{#1}}}}
\usepackage{graphicx}
\makeatletter
\def\maxwidth{\ifdim\Gin@nat@width>\linewidth\linewidth\else\Gin@nat@width\fi}
\def\maxheight{\ifdim\Gin@nat@height>\textheight\textheight\else\Gin@nat@height\fi}
\makeatother
% Scale images if necessary, so that they will not overflow the page
% margins by default, and it is still possible to overwrite the defaults
% using explicit options in \includegraphics[width, height, ...]{}
\setkeys{Gin}{width=\maxwidth,height=\maxheight,keepaspectratio}
% Set default figure placement to htbp
\makeatletter
\def\fps@figure{htbp}
\makeatother
\setlength{\emergencystretch}{3em} % prevent overfull lines
\providecommand{\tightlist}{%
  \setlength{\itemsep}{0pt}\setlength{\parskip}{0pt}}
\setcounter{secnumdepth}{-\maxdimen} % remove section numbering
\ifLuaTeX
  \usepackage{selnolig}  % disable illegal ligatures
\fi
\usepackage{bookmark}
\IfFileExists{xurl.sty}{\usepackage{xurl}}{} % add URL line breaks if available
\urlstyle{same}
\hypersetup{
  pdftitle={Sediment Deposition Rates Across Distances and Plant Dimensions In Two Marshes of San Pablo Bay, CA},
  pdfauthor={Savannah K. Miller},
  hidelinks,
  pdfcreator={LaTeX via pandoc}}

\title{Sediment Deposition Rates Across Distances and Plant Dimensions
In Two Marshes of San Pablo Bay, CA}
\author{Savannah K. Miller}
\date{2025-03-21}

\begin{document}
\maketitle

\paragraph{\texorpdfstring{Dataset Card Source:
(\url{https://github.com/Smillersfsu/DepositionSFB/blob/0a6b8b2033b76344b3b1574f88ab81aa244e5210/DatasetCard.md})}{Dataset Card Source: (https://github.com/Smillersfsu/DepositionSFB/blob/0a6b8b2033b76344b3b1574f88ab81aa244e5210/DatasetCard.md)}}\label{dataset-card-source-httpsgithub.comsmillersfsudepositionsfbblob0a6b8b2033b76344b3b1574f88ab81aa244e5210datasetcard.md}

\subsection{Background}\label{background}

Sediment is the structural foundation of the salt marsh ecosystem where
endemic species live and acquire resources (Levin et al.~2001; Janousek
et al.~2017). The salt marsh ecosystem is host to many endemic species
that are specially adapted to the conditions of the salt marsh (Dunson
and Travis 1994; Schile et al.~2011; Moffett, Robinson, and Gorelick
2010; Rankin et al.~2023), and the very sediment they walk and live
within is in short supply (Schoellhamer and Marineau, n.d.). Salt
marshes play strong roles in dampening impacts of atmospheric rivers and
extreme weather events, offering a protection to coastal
communities(Smolders et al.~2015; Castagno et al.~2022; Lee and Nepf
2024; Taylor-Burns et al.~2024). Already, there has been a long human
history of disturbance to existing salt marshes, which puts ecological
and human communities at severe risk(Gedan, Silliman, and Bertness
2009). Under these circumstances, the few salt marshes left are at risk
of elimination due to drowning if action is not taken and the benefits
of this ecosystem could be forever lost.(Endris et al.~2024).

Across the state, salt marshes are decreasing in number and area due to
a variety of factors (Endris et al.~2024). This unfortunate circumstance
paired with changing water levels along the coastline increases the
sense of urgency to understand, conserve, and restore California's salt
marshes. To assist in raising the elevation of the salt marshes to adapt
to natural hazards, we need to understand the fundamental processes of
how sediment accumulates within the salt marsh over a brief period
(short-term deposition) (Fagherazzi 2013; Houttuijn Bloemendaal et
al.~2021; Vandenbruwaene et al.~2011). In this study, I will measure
short-term deposition rates across marsh edges of multiple marshes
within a single estuary in San Pablo Bay of the San Francisco Bay
Estuary and comparing those rates to measurements of the surrounding
marsh landscape. This study will provide information to ongoing
restoration efforts of salt marshes in San Pablo Bay.(Callaway 2024;
University of San Francisco et al.~2011; K. M. Thorne et al.~2019;
Haltiner et al.~1996) My study provides novel insights for land managers
to characterize their marshes edges to adapt to natural hazards and
enhance community resiliency (Reed et al.~1999).

In this study, I will measure short-term deposition rates on multiple
marshes within the San Francisco Bay Estuary and will compare those
rates to measurements of the surrounding marsh landscape. This study
builds on the work of Karen Thorne and U.S. Geological Survey (2025;
Inter-and Intra-Annual Sediment Dynamics in Two Tidal Marshes:
Deposition, Accretion, and Vegetation Data - ScienceBase-Catalog) to
understand differences in sediment deposition on marshes with different
edge characteristics, vegetation densities and geographic settings.
Moreover, I will measure variables such as elevation, vegetation
structure, and observations of marsh edge traits and comparing to rates
of short-term deposition measured to understand patterns between
variables. Other environmental variables such as season at time of
collection and tide level at time of collection will be recorded and
considered in analyses.

\subsection{Research Objective}\label{research-objective}

This study will measure short-term deposition rates of sediment across
the exposed (bay-adjacent) salt marshes in the northern part of the San
Francisco Bay Estuary of the San Pablo Bay. I will compare these rates
to biological structures of each salt marsh.

\subsubsection{Research Aim:}\label{research-aim}

Marsh characteristics of the vegetation also are one of the first
interactions with the waves bay ward and along the marsh edge. These
vegetation characteristics will be measured at each marsh and compared
with the rates of sediment deposition measured over the course of a
year's spring-neap tidal cycles.

\subsubsection{Expected Outcome:}\label{expected-outcome}

Deposition rates vary significantly with characteristics of the marsh
vegetation, specifically maximum height.

\subsubsection{Null Hypothesis:}\label{null-hypothesis}

Deposition rates of sediment do not vary with differences in height of
Salicornia pacificia (Pickleweed).

\subsubsection{Alternate hypothesis}\label{alternate-hypothesis}

Deposition rates of sediment do vary with differences in height of
Salicornia pacifica (Pickleweed).

\subsection{Data analysis Plan}\label{data-analysis-plan}

First, I played with both the vegetation and sediment data sets to
understand patterns that each of those explanatory variables had with
the landscape and abiotic factors, such as distance from the water
source (``Distance''), weather factors (``Season''), time factors
(``Year'' and ``Month''), and location (``Site'').

As the sample distribution for the sediment flux measurements was not
normally distributed, which was my main independent variable, I did not
run a two-sampled t-test.

I ran a general linear mixed model with both sites data and sediment
deposition versus the maximum plant height. I followed this by running a
general linear mixed model for each site with those same explanatory
variables to compare the AIC values and find the best fit, with distance
from the marsh edge as an additional predictor. Within these models, I
accounted for random effects such as ``Year'', ``Season'' and ``Site''.

\begin{Shaded}
\begin{Highlighting}[]
\FunctionTok{library}\NormalTok{(ggplot2)}
\end{Highlighting}
\end{Shaded}

\begin{verbatim}
## Warning: package 'ggplot2' was built under R version 4.4.3
\end{verbatim}

\begin{Shaded}
\begin{Highlighting}[]
\FunctionTok{library}\NormalTok{(tidyverse)}
\end{Highlighting}
\end{Shaded}

\begin{verbatim}
## -- Attaching core tidyverse packages ------------------------ tidyverse 2.0.0 --
## v dplyr     1.1.4     v readr     2.1.5
## v forcats   1.0.0     v stringr   1.5.1
## v lubridate 1.9.4     v tibble    3.2.1
## v purrr     1.0.2     v tidyr     1.3.1
## -- Conflicts ------------------------------------------ tidyverse_conflicts() --
## x dplyr::filter() masks stats::filter()
## x dplyr::lag()    masks stats::lag()
## i Use the conflicted package (<http://conflicted.r-lib.org/>) to force all conflicts to become errors
\end{verbatim}

\begin{Shaded}
\begin{Highlighting}[]
\FunctionTok{library}\NormalTok{(Rmisc)}
\end{Highlighting}
\end{Shaded}

\begin{verbatim}
## Warning: package 'Rmisc' was built under R version 4.4.3
\end{verbatim}

\begin{verbatim}
## Loading required package: lattice
## Loading required package: plyr
## ------------------------------------------------------------------------------
## You have loaded plyr after dplyr - this is likely to cause problems.
## If you need functions from both plyr and dplyr, please load plyr first, then dplyr:
## library(plyr); library(dplyr)
## ------------------------------------------------------------------------------
## 
## Attaching package: 'plyr'
## 
## The following objects are masked from 'package:dplyr':
## 
##     arrange, count, desc, failwith, id, mutate, rename, summarise,
##     summarize
## 
## The following object is masked from 'package:purrr':
## 
##     compact
\end{verbatim}

\begin{Shaded}
\begin{Highlighting}[]
\FunctionTok{library}\NormalTok{(rstatix)}
\end{Highlighting}
\end{Shaded}

\begin{verbatim}
## Warning: package 'rstatix' was built under R version 4.4.3
\end{verbatim}

\begin{verbatim}
## 
## Attaching package: 'rstatix'
## 
## The following objects are masked from 'package:plyr':
## 
##     desc, mutate
## 
## The following object is masked from 'package:stats':
## 
##     filter
\end{verbatim}

\begin{Shaded}
\begin{Highlighting}[]
\FunctionTok{library}\NormalTok{(tidyr)}
\FunctionTok{library}\NormalTok{(data.table)}
\end{Highlighting}
\end{Shaded}

\begin{verbatim}
## Warning: package 'data.table' was built under R version 4.4.3
\end{verbatim}

\begin{verbatim}
## 
## Attaching package: 'data.table'
## 
## The following objects are masked from 'package:lubridate':
## 
##     hour, isoweek, mday, minute, month, quarter, second, wday, week,
##     yday, year
## 
## The following objects are masked from 'package:dplyr':
## 
##     between, first, last
## 
## The following object is masked from 'package:purrr':
## 
##     transpose
\end{verbatim}

\begin{Shaded}
\begin{Highlighting}[]
\FunctionTok{library}\NormalTok{(dplyr)}
\FunctionTok{library}\NormalTok{(lme4)}
\end{Highlighting}
\end{Shaded}

\begin{verbatim}
## Warning: package 'lme4' was built under R version 4.4.3
\end{verbatim}

\begin{verbatim}
## Loading required package: Matrix
## 
## Attaching package: 'Matrix'
## 
## The following objects are masked from 'package:tidyr':
## 
##     expand, pack, unpack
\end{verbatim}

\begin{Shaded}
\begin{Highlighting}[]
\FunctionTok{library}\NormalTok{(stringr)}
\end{Highlighting}
\end{Shaded}

\begin{Shaded}
\begin{Highlighting}[]
\NormalTok{knitr}\SpecialCharTok{::}\NormalTok{opts\_knit}\SpecialCharTok{$}\FunctionTok{set}\NormalTok{(}\AttributeTok{root.dir =} \StringTok{\textquotesingle{}C:/Users/savan/OneDrive/Documents/Code/GitHub/DepositionSFB/data\textquotesingle{}}\NormalTok{)}
\end{Highlighting}
\end{Shaded}

\begin{Shaded}
\begin{Highlighting}[]
\CommentTok{\# load data}
\NormalTok{dataset\_sed }\OtherTok{\textless{}{-}} \FunctionTok{read.csv}\NormalTok{(}\StringTok{"C:/Users/savan/OneDrive/Documents/Code/GitHub/DepositionSFB/data/SFB\_sediment\_pad\_data (project).csv"}\NormalTok{) }

\NormalTok{dataset\_veg }\OtherTok{\textless{}{-}} \FunctionTok{read.csv}\NormalTok{(}\StringTok{"C:/Users/savan/OneDrive/Documents/Code/GitHub/DepositionSFB/data/SFB\_1m\_vegetation\_data (project).csv"}\NormalTok{) }

\NormalTok{sed }\OtherTok{\textless{}{-}}\NormalTok{ dataset\_sed }\SpecialCharTok{\%\textgreater{}\%} \FunctionTok{mutate}\NormalTok{(}\AttributeTok{FluxAvgRep =} \FunctionTok{ifelse}\NormalTok{(}\FunctionTok{is.na}\NormalTok{(FluxAvgRep), }\DecValTok{0}\NormalTok{, FluxAvgRep))}

\NormalTok{sed }\OtherTok{\textless{}{-}}\NormalTok{ dataset\_sed }\SpecialCharTok{\%\textgreater{}\%}
  \FunctionTok{select}\NormalTok{(}\SpecialCharTok{{-}}\NormalTok{Date.removed.from.field, }\SpecialCharTok{{-}}\NormalTok{Date.placed.in.field)}

\NormalTok{sed }\OtherTok{\textless{}{-}}\NormalTok{ sed }\SpecialCharTok{\%\textgreater{}\%} \FunctionTok{mutate}\NormalTok{(}\AttributeTok{FluxAvgRep =} \FunctionTok{ifelse}\NormalTok{(}\FunctionTok{is.na}\NormalTok{(FluxAvgRep), }\DecValTok{0}\NormalTok{, FluxAvgRep))}

\CommentTok{\#sed$Plot.ID \textless{}{-} paste(sed$Transect, sed$Distance)}
\CommentTok{\#veg$Plot.ID \textless{}{-} paste(veg$Transect, veg$Distance)}
\NormalTok{sed}\SpecialCharTok{$}\NormalTok{Season }\OtherTok{\textless{}{-}} \FunctionTok{str\_replace\_all}\NormalTok{(sed}\SpecialCharTok{$}\NormalTok{Season, }\StringTok{" "}\NormalTok{, }\StringTok{""}\NormalTok{)}
\NormalTok{sed}\SpecialCharTok{$}\NormalTok{Distance }\OtherTok{\textless{}{-}} \FunctionTok{str\_replace\_all}\NormalTok{(sed}\SpecialCharTok{$}\NormalTok{Distance, }\StringTok{" "}\NormalTok{, }\StringTok{""}\NormalTok{)}
\NormalTok{sed}\SpecialCharTok{$}\NormalTok{Transect }\OtherTok{\textless{}{-}} \FunctionTok{str\_replace\_all}\NormalTok{(sed}\SpecialCharTok{$}\NormalTok{Transect, }\StringTok{" "}\NormalTok{, }\StringTok{""}\NormalTok{)}
\NormalTok{sed}\SpecialCharTok{$}\NormalTok{Year }\OtherTok{\textless{}{-}} \FunctionTok{str\_replace\_all}\NormalTok{(sed}\SpecialCharTok{$}\NormalTok{Year, }\StringTok{" "}\NormalTok{, }\StringTok{""}\NormalTok{)}
\NormalTok{sed}\SpecialCharTok{$}\NormalTok{Month }\OtherTok{\textless{}{-}} \FunctionTok{str\_replace\_all}\NormalTok{(sed}\SpecialCharTok{$}\NormalTok{Month, }\StringTok{" "}\NormalTok{, }\StringTok{""}\NormalTok{)}
\NormalTok{sed}\SpecialCharTok{$}\NormalTok{FluxAvgRep }\OtherTok{\textless{}{-}} \FunctionTok{str\_replace\_all}\NormalTok{(sed}\SpecialCharTok{$}\NormalTok{FluxAvgRep, }\StringTok{" "}\NormalTok{, }\StringTok{""}\NormalTok{)}
\CommentTok{\#sed$Plot.ID \textless{}{-} str\_replace\_all(sed$Plot.ID, " ", "")}
\NormalTok{sed}\SpecialCharTok{$}\NormalTok{FluxAvgRep }\OtherTok{\textless{}{-}}\FunctionTok{as.numeric}\NormalTok{(sed}\SpecialCharTok{$}\NormalTok{FluxAvgRep)}
\end{Highlighting}
\end{Shaded}

\begin{verbatim}
## Warning: NAs introduced by coercion
\end{verbatim}

\begin{Shaded}
\begin{Highlighting}[]
\NormalTok{sed}\SpecialCharTok{$}\NormalTok{FluxAvgRep }\OtherTok{\textless{}{-}} \FunctionTok{round}\NormalTok{(sed}\SpecialCharTok{$}\NormalTok{FluxAvgRep)}
\NormalTok{sed}\SpecialCharTok{$}\NormalTok{Year }\OtherTok{\textless{}{-}} \FunctionTok{as.character}\NormalTok{(sed}\SpecialCharTok{$}\NormalTok{Year)}
\NormalTok{sed}\SpecialCharTok{$}\NormalTok{Distance }\OtherTok{\textless{}{-}}\FunctionTok{as.numeric}\NormalTok{(sed}\SpecialCharTok{$}\NormalTok{Distance)}


\NormalTok{veg }\OtherTok{\textless{}{-}}\NormalTok{ dataset\_veg }\SpecialCharTok{\%\textgreater{}\%} \FunctionTok{mutate}\NormalTok{(}\AttributeTok{Avg.Ht =} \FunctionTok{ifelse}\NormalTok{(}\FunctionTok{is.na}\NormalTok{(Avg.Ht), }\DecValTok{0}\NormalTok{, Avg.Ht))}
\NormalTok{veg }\OtherTok{\textless{}{-}}\NormalTok{ dataset\_veg }\SpecialCharTok{\%\textgreater{}\%} \FunctionTok{mutate}\NormalTok{(}\AttributeTok{Max.Ht =} \FunctionTok{ifelse}\NormalTok{(}\FunctionTok{is.na}\NormalTok{(Max.Ht), }\DecValTok{0}\NormalTok{, Max.Ht))}
\NormalTok{veg}\SpecialCharTok{$}\NormalTok{Season }\OtherTok{\textless{}{-}} \FunctionTok{str\_replace\_all}\NormalTok{(veg}\SpecialCharTok{$}\NormalTok{Season, }\StringTok{" "}\NormalTok{, }\StringTok{""}\NormalTok{)}
\NormalTok{veg}\SpecialCharTok{$}\NormalTok{Distance }\OtherTok{\textless{}{-}} \FunctionTok{str\_replace\_all}\NormalTok{(veg}\SpecialCharTok{$}\NormalTok{Distance, }\StringTok{" "}\NormalTok{, }\StringTok{""}\NormalTok{)}
\NormalTok{veg}\SpecialCharTok{$}\NormalTok{Transect }\OtherTok{\textless{}{-}} \FunctionTok{str\_replace\_all}\NormalTok{(veg}\SpecialCharTok{$}\NormalTok{Transect, }\StringTok{" "}\NormalTok{, }\StringTok{""}\NormalTok{)}
\NormalTok{veg}\SpecialCharTok{$}\NormalTok{Year }\OtherTok{\textless{}{-}} \FunctionTok{str\_replace\_all}\NormalTok{(veg}\SpecialCharTok{$}\NormalTok{Year, }\StringTok{" "}\NormalTok{, }\StringTok{""}\NormalTok{)}
\NormalTok{veg}\SpecialCharTok{$}\NormalTok{Month }\OtherTok{\textless{}{-}} \FunctionTok{str\_replace\_all}\NormalTok{(veg}\SpecialCharTok{$}\NormalTok{Month, }\StringTok{" "}\NormalTok{, }\StringTok{""}\NormalTok{)}
\NormalTok{veg}\SpecialCharTok{$}\NormalTok{Year }\OtherTok{\textless{}{-}} \FunctionTok{as.character}\NormalTok{(veg}\SpecialCharTok{$}\NormalTok{Year)}
\NormalTok{veg}\SpecialCharTok{$}\NormalTok{Cover }\OtherTok{\textless{}{-}}\FunctionTok{as.numeric}\NormalTok{(veg}\SpecialCharTok{$}\NormalTok{Cover)}
\NormalTok{veg}\SpecialCharTok{$}\NormalTok{Distance }\OtherTok{\textless{}{-}}\FunctionTok{as.numeric}\NormalTok{(veg}\SpecialCharTok{$}\NormalTok{Distance)}

\FunctionTok{str}\NormalTok{(sed)}
\end{Highlighting}
\end{Shaded}

\begin{verbatim}
## 'data.frame':    183 obs. of  7 variables:
##  $ Site      : chr  "Corte Madera" "Corte Madera" "Corte Madera" "Corte Madera" ...
##  $ Transect  : chr  "A" "A" "A" "A" ...
##  $ Distance  : num  2 6 12 24 48 0.5 2 6 12 24 ...
##  $ Month     : chr  "AUGUST" "AUGUST" "AUGUST" "AUGUST" ...
##  $ Season    : chr  "Summer" "Summer" "Summer" "Summer" ...
##  $ Year      : chr  "2022" "2022" "2022" "2022" ...
##  $ FluxAvgRep: num  4 2 8 18 6 0 3 4 8 24 ...
\end{verbatim}

\begin{Shaded}
\begin{Highlighting}[]
\FunctionTok{str}\NormalTok{(veg)}
\end{Highlighting}
\end{Shaded}

\begin{verbatim}
## 'data.frame':    115 obs. of  11 variables:
##  $ Month   : chr  "AUGUST" "AUGUST" "AUGUST" "AUGUST" ...
##  $ Season  : chr  "Summer" "Summer" "Summer" "Summer" ...
##  $ Year    : chr  "2022" "2022" "2022" "2022" ...
##  $ Date    : chr  "8/24/2022" "8/24/2022" "8/24/2022" "8/24/2022" ...
##  $ Site    : chr  "Corte Madera" "Corte Madera" "Corte Madera" "Corte Madera" ...
##  $ Transect: chr  "A" "A" "A" "A" ...
##  $ Distance: num  0.5 12 2 24 6 0.5 12 24 48 6 ...
##  $ Species : chr  "SAPA" "SAPA" "SAPA" "SAPA" ...
##  $ Avg.Ht  : num  44 28 26 30 31 26 28 27 21 22 ...
##  $ Max.Ht  : num  54 41 37 45 39 50 35 37 37 38 ...
##  $ Cover   : num  60 10 30 50 5 55 45 15 50 30 ...
\end{verbatim}

\begin{Shaded}
\begin{Highlighting}[]
\NormalTok{sed\_mean }\OtherTok{\textless{}{-}} \FunctionTok{mean}\NormalTok{(sed}\SpecialCharTok{$}\NormalTok{FluxAvgRep, }\AttributeTok{na.rm =} \ConstantTok{TRUE}\NormalTok{)}
\NormalTok{sed\_mean}
\end{Highlighting}
\end{Shaded}

\begin{verbatim}
## [1] 15.91758
\end{verbatim}

\begin{Shaded}
\begin{Highlighting}[]
\NormalTok{sed\_max  }\OtherTok{\textless{}{-}} \FunctionTok{max}\NormalTok{(sed}\SpecialCharTok{$}\NormalTok{FluxAvgRep, }\AttributeTok{na.rm =} \ConstantTok{TRUE}\NormalTok{)}
\NormalTok{sed\_max}
\end{Highlighting}
\end{Shaded}

\begin{verbatim}
## [1] 356
\end{verbatim}

\begin{Shaded}
\begin{Highlighting}[]
\NormalTok{sed\_min  }\OtherTok{\textless{}{-}} \FunctionTok{min}\NormalTok{(sed}\SpecialCharTok{$}\NormalTok{FluxAvgRep, }\AttributeTok{na.rm =} \ConstantTok{TRUE}\NormalTok{)}
\NormalTok{sed\_min}
\end{Highlighting}
\end{Shaded}

\begin{verbatim}
## [1] 0
\end{verbatim}

\begin{Shaded}
\begin{Highlighting}[]
\NormalTok{sed\_sd }\OtherTok{\textless{}{-}}\FunctionTok{sd}\NormalTok{(sed}\SpecialCharTok{$}\NormalTok{FluxAvgRep, }\AttributeTok{na.rm =}\ConstantTok{TRUE}\NormalTok{)}
\NormalTok{sed\_sd}
\end{Highlighting}
\end{Shaded}

\begin{verbatim}
## [1] 41.69448
\end{verbatim}

\begin{Shaded}
\begin{Highlighting}[]
\NormalTok{sed\_SAPA }\OtherTok{\textless{}{-}}\NormalTok{ sed }\SpecialCharTok{\%\textgreater{}\%} \FunctionTok{left\_join}\NormalTok{(veg, }\AttributeTok{by =} \FunctionTok{c}\NormalTok{(}\StringTok{"Site"}\NormalTok{, }\StringTok{"Season"}\NormalTok{, }\StringTok{"Distance"}\NormalTok{, }\StringTok{"Transect"}\NormalTok{, }\StringTok{"Year"}\NormalTok{, }\StringTok{"Month"}\NormalTok{))}
\NormalTok{sed\_SAPA }\OtherTok{\textless{}{-}}\NormalTok{ sed\_SAPA }\SpecialCharTok{\%\textgreater{}\%}
  \FunctionTok{mutate}\NormalTok{(}\AttributeTok{FluxAvgRep =} \FunctionTok{ifelse}\NormalTok{(}\FunctionTok{is.na}\NormalTok{(FluxAvgRep), }\DecValTok{0}\NormalTok{, FluxAvgRep),}
        \AttributeTok{Species =} \FunctionTok{ifelse}\NormalTok{(}\FunctionTok{is.na}\NormalTok{(Species), }\DecValTok{0}\NormalTok{, Species), }
        \AttributeTok{Avg.Ht =} \FunctionTok{ifelse}\NormalTok{(}\FunctionTok{is.na}\NormalTok{(Avg.Ht), }\DecValTok{0}\NormalTok{, Avg.Ht),}
        \AttributeTok{Max.Ht =} \FunctionTok{ifelse}\NormalTok{(}\FunctionTok{is.na}\NormalTok{(Max.Ht), }\DecValTok{0}\NormalTok{, Max.Ht), }
        \AttributeTok{Cover =} \FunctionTok{ifelse}\NormalTok{(}\FunctionTok{is.na}\NormalTok{(Cover), }\DecValTok{0}\NormalTok{, Cover))}
\NormalTok{sed\_SAPA }\OtherTok{\textless{}{-}}\NormalTok{ sed\_SAPA }\SpecialCharTok{\%\textgreater{}\%}
  \FunctionTok{select}\NormalTok{(}\SpecialCharTok{{-}}\NormalTok{Date)}
\end{Highlighting}
\end{Shaded}

\begin{Shaded}
\begin{Highlighting}[]
\NormalTok{SAPA\_dist\_mx }\OtherTok{\textless{}{-}} \FunctionTok{ggplot}\NormalTok{(}\AttributeTok{data=}\NormalTok{sed\_SAPA, }\FunctionTok{aes}\NormalTok{(}\AttributeTok{x=}\NormalTok{Distance, }\AttributeTok{y=}\NormalTok{Max.Ht, }\AttributeTok{color=}\NormalTok{Site)) }\SpecialCharTok{+} \FunctionTok{geom\_point}\NormalTok{() }\SpecialCharTok{+}  \FunctionTok{facet\_wrap}\NormalTok{(}\SpecialCharTok{\textasciitilde{}}\NormalTok{ Season)}
\NormalTok{SAPA\_dist\_mx}
\end{Highlighting}
\end{Shaded}

\includegraphics{Independent-Project-Final-Project-Savannah-Miller_files/figure-latex/visualizing relationships of the vegetation data to explanatory variables-1.pdf}

\begin{Shaded}
\begin{Highlighting}[]
\NormalTok{SAPA\_dist\_avg }\OtherTok{\textless{}{-}} \FunctionTok{ggplot}\NormalTok{(}\AttributeTok{data=}\NormalTok{sed\_SAPA, }\FunctionTok{aes}\NormalTok{(}\AttributeTok{x=}\NormalTok{Distance, }\AttributeTok{y=}\NormalTok{Avg.Ht, }\AttributeTok{color=}\NormalTok{Site)) }\SpecialCharTok{+} \FunctionTok{geom\_point}\NormalTok{() }\SpecialCharTok{+} \FunctionTok{facet\_wrap}\NormalTok{(}\SpecialCharTok{\textasciitilde{}}\NormalTok{ Season)}
\NormalTok{SAPA\_dist\_avg}
\end{Highlighting}
\end{Shaded}

\includegraphics{Independent-Project-Final-Project-Savannah-Miller_files/figure-latex/visualizing relationships of the vegetation data to explanatory variables-2.pdf}

\begin{Shaded}
\begin{Highlighting}[]
\NormalTok{SAPA\_dist\_c }\OtherTok{\textless{}{-}} \FunctionTok{ggplot}\NormalTok{(}\AttributeTok{data=}\NormalTok{sed\_SAPA, }\FunctionTok{aes}\NormalTok{(}\AttributeTok{x=}\NormalTok{Distance, }\AttributeTok{y=}\NormalTok{Cover, }\AttributeTok{color=}\NormalTok{Site)) }\SpecialCharTok{+} \FunctionTok{geom\_point}\NormalTok{() }\SpecialCharTok{+}\FunctionTok{facet\_wrap}\NormalTok{(}\SpecialCharTok{\textasciitilde{}}\NormalTok{ Season)}
\NormalTok{SAPA\_dist\_c}
\end{Highlighting}
\end{Shaded}

\includegraphics{Independent-Project-Final-Project-Savannah-Miller_files/figure-latex/visualizing relationships of the vegetation data to explanatory variables-3.pdf}

\begin{Shaded}
\begin{Highlighting}[]
\NormalTok{sed\_space }\OtherTok{\textless{}{-}}\FunctionTok{ggplot}\NormalTok{(}\AttributeTok{data=}\NormalTok{sed\_SAPA, }\FunctionTok{aes}\NormalTok{(}\AttributeTok{x=}\NormalTok{Distance, }\AttributeTok{y=}\NormalTok{FluxAvgRep, }\AttributeTok{color=}\NormalTok{Site)) }\SpecialCharTok{+}\FunctionTok{geom\_point}\NormalTok{() }\SpecialCharTok{+} \FunctionTok{geom\_jitter}\NormalTok{(}\AttributeTok{width =} \FloatTok{0.2}\NormalTok{, }\AttributeTok{height =} \FloatTok{0.2}\NormalTok{, }\AttributeTok{alpha =} \FloatTok{0.6}\NormalTok{) }\SpecialCharTok{+} \FunctionTok{theme\_minimal}\NormalTok{() }\SpecialCharTok{+}
 \FunctionTok{coord\_cartesian}\NormalTok{(}\AttributeTok{ylim =} \FunctionTok{c}\NormalTok{(}\DecValTok{0}\NormalTok{, }\DecValTok{357}\NormalTok{)) }\SpecialCharTok{+} \FunctionTok{facet\_wrap}\NormalTok{(}\SpecialCharTok{\textasciitilde{}}\NormalTok{ Season)}
\NormalTok{sed\_space}
\end{Highlighting}
\end{Shaded}

\includegraphics{Independent-Project-Final-Project-Savannah-Miller_files/figure-latex/visualizing relationships of the sediment data to explanatory variables-1.pdf}

\begin{Shaded}
\begin{Highlighting}[]
\NormalTok{sed\_time }\OtherTok{\textless{}{-}}\FunctionTok{ggplot}\NormalTok{(}\AttributeTok{data=}\NormalTok{sed\_SAPA, }\FunctionTok{aes}\NormalTok{(}\AttributeTok{x=}\NormalTok{Season, }\AttributeTok{y=}\NormalTok{FluxAvgRep, }\AttributeTok{color=}\NormalTok{Site)) }\SpecialCharTok{+} \FunctionTok{geom\_point}\NormalTok{() }\SpecialCharTok{+} \FunctionTok{geom\_jitter}\NormalTok{(}\AttributeTok{width =} \FloatTok{0.2}\NormalTok{, }\AttributeTok{height =} \FloatTok{0.2}\NormalTok{, }\AttributeTok{alpha =} \FloatTok{0.6}\NormalTok{) }\SpecialCharTok{+} \FunctionTok{theme\_minimal}\NormalTok{() }\SpecialCharTok{+}
 \FunctionTok{coord\_cartesian}\NormalTok{(}\AttributeTok{ylim =} \FunctionTok{c}\NormalTok{(}\DecValTok{0}\NormalTok{, }\DecValTok{357}\NormalTok{))}
\NormalTok{sed\_time}
\end{Highlighting}
\end{Shaded}

\includegraphics{Independent-Project-Final-Project-Savannah-Miller_files/figure-latex/visualizing relationships of the sediment data to explanatory variables-2.pdf}

\begin{Shaded}
\begin{Highlighting}[]
\NormalTok{sed\_year }\OtherTok{\textless{}{-}} \FunctionTok{ggplot}\NormalTok{(sed\_SAPA, }\FunctionTok{aes}\NormalTok{(}\AttributeTok{x=}\NormalTok{ Season, }\AttributeTok{y=}\NormalTok{ FluxAvgRep, }\AttributeTok{color=}\NormalTok{ Site)) }\SpecialCharTok{+}\FunctionTok{geom\_point}\NormalTok{() }\SpecialCharTok{+} \FunctionTok{facet\_wrap}\NormalTok{(}\SpecialCharTok{\textasciitilde{}}\NormalTok{Year)}
\NormalTok{sed\_year}
\end{Highlighting}
\end{Shaded}

\includegraphics{Independent-Project-Final-Project-Savannah-Miller_files/figure-latex/visualizing relationships of the sediment data to explanatory variables-3.pdf}

\begin{Shaded}
\begin{Highlighting}[]
\NormalTok{sed\_season\_site }\OtherTok{=} \FunctionTok{ggplot}\NormalTok{(sed, }\FunctionTok{aes}\NormalTok{(}\AttributeTok{x =}\NormalTok{ Season, }\AttributeTok{y =}\NormalTok{ FluxAvgRep, }\AttributeTok{fill =}\NormalTok{ Site)) }\SpecialCharTok{+}
  \FunctionTok{facet\_wrap}\NormalTok{(}\SpecialCharTok{\textasciitilde{}}\NormalTok{Distance) }\SpecialCharTok{+}
  \FunctionTok{geom\_col}\NormalTok{(}\AttributeTok{position =} \StringTok{"dodge"}\NormalTok{) }\SpecialCharTok{+}
  \FunctionTok{scale\_y\_log10}\NormalTok{() }\SpecialCharTok{+}
  \FunctionTok{labs}\NormalTok{(}\AttributeTok{x =} \StringTok{"Season"}\NormalTok{, }\AttributeTok{y =} \StringTok{"Log Flux of Sediment Deposition"}\NormalTok{) }
\NormalTok{sed\_season\_site}
\end{Highlighting}
\end{Shaded}

\begin{verbatim}
## Warning in scale_y_log10(): log-10 transformation introduced infinite values.
\end{verbatim}

\begin{verbatim}
## Warning: Removed 1 row containing missing values or values outside the scale range
## (`geom_col()`).
\end{verbatim}

\includegraphics{Independent-Project-Final-Project-Savannah-Miller_files/figure-latex/visualizing relationships of the sediment data to explanatory variables-4.pdf}

\begin{Shaded}
\begin{Highlighting}[]
\NormalTok{sed\_Siteseason }\OtherTok{\textless{}{-}} \FunctionTok{ggplot}\NormalTok{(sed\_SAPA, }\FunctionTok{aes}\NormalTok{(}\AttributeTok{x=}\NormalTok{ Site, }\AttributeTok{y=}\NormalTok{FluxAvgRep, }\AttributeTok{color =}\NormalTok{ Season)) }\SpecialCharTok{+}\FunctionTok{geom\_point}\NormalTok{() }\SpecialCharTok{+} \FunctionTok{geom\_boxplot}\NormalTok{()}
\NormalTok{sed\_Siteseason}
\end{Highlighting}
\end{Shaded}

\includegraphics{Independent-Project-Final-Project-Savannah-Miller_files/figure-latex/visualizing relationships of the sediment data to explanatory variables-5.pdf}

\begin{Shaded}
\begin{Highlighting}[]
\NormalTok{sed\_season }\OtherTok{=} \FunctionTok{ggplot}\NormalTok{(sed, }\FunctionTok{aes}\NormalTok{(}\AttributeTok{x =}\NormalTok{ Site, }\AttributeTok{y =}\NormalTok{ FluxAvgRep, }\AttributeTok{fill =}\NormalTok{ Season)) }\SpecialCharTok{+} 
  \FunctionTok{geom\_col}\NormalTok{(}\AttributeTok{position =} \StringTok{"dodge"}\NormalTok{) }\SpecialCharTok{+}
  \FunctionTok{scale\_y\_log10}\NormalTok{() }\SpecialCharTok{+}
  \FunctionTok{labs}\NormalTok{(}\AttributeTok{x =} \StringTok{"Site"}\NormalTok{, }\AttributeTok{y =} \StringTok{"Log Flux of Sediment Deposition"}\NormalTok{) }
\NormalTok{sed\_season}
\end{Highlighting}
\end{Shaded}

\begin{verbatim}
## Warning in scale_y_log10(): log-10 transformation introduced infinite values.
## Removed 1 row containing missing values or values outside the scale range
## (`geom_col()`).
\end{verbatim}

\includegraphics{Independent-Project-Final-Project-Savannah-Miller_files/figure-latex/visualizing relationships of the sediment data to explanatory variables-6.pdf}

\begin{Shaded}
\begin{Highlighting}[]
\NormalTok{sed\_Dist }\OtherTok{=} \FunctionTok{ggplot}\NormalTok{(sed, }\FunctionTok{aes}\NormalTok{(}\AttributeTok{x =}\NormalTok{ Distance, }\AttributeTok{y =}\NormalTok{ FluxAvgRep, }\AttributeTok{color =}\NormalTok{ Site)) }\SpecialCharTok{+} 
  \FunctionTok{geom\_point}\NormalTok{(}\FunctionTok{aes}\NormalTok{(}\AttributeTok{x =}\NormalTok{ Distance, }\AttributeTok{y =}\NormalTok{ FluxAvgRep), }\AttributeTok{data =}\NormalTok{ sed, }\AttributeTok{position =} \FunctionTok{position\_jitter}\NormalTok{(}\AttributeTok{w =} \FloatTok{0.05}\NormalTok{, }\AttributeTok{h =} \DecValTok{0}\NormalTok{)) }\SpecialCharTok{+}
  \CommentTok{\#scale\_y\_log10() + }
   \FunctionTok{geom\_smooth}\NormalTok{(}\AttributeTok{method =} \StringTok{"lm"}\NormalTok{, }\AttributeTok{se =} \ConstantTok{FALSE}\NormalTok{) }\SpecialCharTok{+}
  \FunctionTok{labs}\NormalTok{(}\AttributeTok{x =} \StringTok{"Distance"}\NormalTok{, }\AttributeTok{y =} \StringTok{"Log Flux of Sediment Deposition"}\NormalTok{, }
       \AttributeTok{title =} \StringTok{"Sediment Deposition Rates Across Distances"}\NormalTok{)}
\NormalTok{sed\_Dist}
\end{Highlighting}
\end{Shaded}

\begin{verbatim}
## `geom_smooth()` using formula = 'y ~ x'
\end{verbatim}

\begin{verbatim}
## Warning: Removed 1 row containing non-finite outside the scale range
## (`stat_smooth()`).
\end{verbatim}

\begin{verbatim}
## Warning: Removed 1 row containing missing values or values outside the scale range
## (`geom_point()`).
\end{verbatim}

\includegraphics{Independent-Project-Final-Project-Savannah-Miller_files/figure-latex/visualizing relationships of the sediment data to explanatory variables-7.pdf}

For visualizing the vegetation data, I used the original veg data set so
that the average wouldn't be swayed by the zero values within the
combined data frame of ``sed\_SAPA'' that accounted for observations of
sediment data collected when vegetation data wasn't consequently
collected.

\begin{Shaded}
\begin{Highlighting}[]
\NormalTok{SAPA\_Season\_mxht }\OtherTok{\textless{}{-}} \FunctionTok{ggplot}\NormalTok{(veg, }\FunctionTok{aes}\NormalTok{(}\AttributeTok{x=}\NormalTok{Max.Ht,}\AttributeTok{fill=}\NormalTok{Season)) }\SpecialCharTok{+} \FunctionTok{geom\_boxplot}\NormalTok{()}
\NormalTok{SAPA\_Season\_mxht}
\end{Highlighting}
\end{Shaded}

\includegraphics{Independent-Project-Final-Project-Savannah-Miller_files/figure-latex/checking the range of data for the vegetation dataset-1.pdf}

\begin{Shaded}
\begin{Highlighting}[]
\NormalTok{SAPA\_Season\_avght }\OtherTok{\textless{}{-}}\FunctionTok{ggplot}\NormalTok{(veg, }\FunctionTok{aes}\NormalTok{(}\AttributeTok{x=}\NormalTok{Avg.Ht,}\AttributeTok{fill=}\NormalTok{Season)) }\SpecialCharTok{+} \FunctionTok{geom\_boxplot}\NormalTok{()}
\NormalTok{SAPA\_Season\_avght}
\end{Highlighting}
\end{Shaded}

\includegraphics{Independent-Project-Final-Project-Savannah-Miller_files/figure-latex/checking the range of data for the vegetation dataset-2.pdf}

\begin{Shaded}
\begin{Highlighting}[]
\NormalTok{SAPA\_Season\_cover }\OtherTok{\textless{}{-}} \FunctionTok{ggplot}\NormalTok{(veg, }\FunctionTok{aes}\NormalTok{(}\AttributeTok{x=}\NormalTok{Cover, }\AttributeTok{fill =}\NormalTok{ Season)) }\SpecialCharTok{+} \FunctionTok{geom\_boxplot}\NormalTok{()}
\NormalTok{SAPA\_Season\_cover}
\end{Highlighting}
\end{Shaded}

\includegraphics{Independent-Project-Final-Project-Savannah-Miller_files/figure-latex/checking the range of data for the vegetation dataset-3.pdf}

\begin{Shaded}
\begin{Highlighting}[]
\NormalTok{SAPA\_Site\_cover }\OtherTok{\textless{}{-}} \FunctionTok{ggplot}\NormalTok{(veg, }\FunctionTok{aes}\NormalTok{(}\AttributeTok{x=}\NormalTok{ Cover, }\AttributeTok{fill =}\NormalTok{ Site)) }\SpecialCharTok{+} \FunctionTok{geom\_boxplot}\NormalTok{()}
\NormalTok{SAPA\_Site\_cover}
\end{Highlighting}
\end{Shaded}

\includegraphics{Independent-Project-Final-Project-Savannah-Miller_files/figure-latex/checking the range of data for the vegetation dataset-4.pdf}

\begin{Shaded}
\begin{Highlighting}[]
\NormalTok{SAPA\_Site\_mxht }\OtherTok{\textless{}{-}} \FunctionTok{ggplot}\NormalTok{(veg, }\FunctionTok{aes}\NormalTok{(}\AttributeTok{x =}\NormalTok{ Max.Ht, }\AttributeTok{fill =}\NormalTok{ Site)) }\SpecialCharTok{+} \FunctionTok{geom\_boxplot}\NormalTok{()}
\NormalTok{SAPA\_Site\_mxht}
\end{Highlighting}
\end{Shaded}

\includegraphics{Independent-Project-Final-Project-Savannah-Miller_files/figure-latex/checking the range of data for the vegetation dataset-5.pdf}

\begin{Shaded}
\begin{Highlighting}[]
\NormalTok{SAPA\_Site\_avght }\OtherTok{\textless{}{-}} \FunctionTok{ggplot}\NormalTok{(veg, }\FunctionTok{aes}\NormalTok{(}\AttributeTok{x =}\NormalTok{ Avg.Ht, }\AttributeTok{fill =}\NormalTok{ Site)) }\SpecialCharTok{+} \FunctionTok{geom\_boxplot}\NormalTok{()}
\NormalTok{SAPA\_Site\_avght}
\end{Highlighting}
\end{Shaded}

\includegraphics{Independent-Project-Final-Project-Savannah-Miller_files/figure-latex/checking the range of data for the vegetation dataset-6.pdf}

\begin{Shaded}
\begin{Highlighting}[]
\CommentTok{\#check the data for normality, Max.Ht }
\NormalTok{SAPA\_Maxht\_Normality }\OtherTok{\textless{}{-}}\FunctionTok{ggplot}\NormalTok{(veg, }\FunctionTok{aes}\NormalTok{(}\AttributeTok{x=}\NormalTok{Max.Ht)) }\SpecialCharTok{+} \FunctionTok{geom\_histogram}\NormalTok{(}\AttributeTok{binwidth =} \FloatTok{0.5}\NormalTok{)}
\NormalTok{SAPA\_Maxht\_Normality}
\end{Highlighting}
\end{Shaded}

\includegraphics{Independent-Project-Final-Project-Savannah-Miller_files/figure-latex/checking the range of data for the vegetation dataset-7.pdf}

\begin{Shaded}
\begin{Highlighting}[]
\CommentTok{\#check the data for normality, Avg.ht  }
\NormalTok{SAPA\_Avght\_Normality }\OtherTok{\textless{}{-}}\FunctionTok{ggplot}\NormalTok{(veg, }\FunctionTok{aes}\NormalTok{(}\AttributeTok{x=}\NormalTok{Avg.Ht)) }\SpecialCharTok{+} \FunctionTok{geom\_histogram}\NormalTok{(}\AttributeTok{binwidth =} \FloatTok{0.5}\NormalTok{)}
\NormalTok{SAPA\_Avght\_Normality}
\end{Highlighting}
\end{Shaded}

\includegraphics{Independent-Project-Final-Project-Savannah-Miller_files/figure-latex/checking the range of data for the vegetation dataset-8.pdf}

\begin{Shaded}
\begin{Highlighting}[]
\CommentTok{\#check the data for normality, Cover }
\NormalTok{SAPA\_Cover\_Normality }\OtherTok{\textless{}{-}}\FunctionTok{ggplot}\NormalTok{(veg, }\FunctionTok{aes}\NormalTok{(}\AttributeTok{x=}\NormalTok{Cover)) }\SpecialCharTok{+} \FunctionTok{geom\_histogram}\NormalTok{(}\AttributeTok{binwidth =} \FloatTok{0.5}\NormalTok{)}
\NormalTok{SAPA\_Cover\_Normality}
\end{Highlighting}
\end{Shaded}

\includegraphics{Independent-Project-Final-Project-Savannah-Miller_files/figure-latex/checking the range of data for the vegetation dataset-9.pdf}

\begin{Shaded}
\begin{Highlighting}[]
\NormalTok{sedvegglm }\OtherTok{\textless{}{-}}\FunctionTok{glm}\NormalTok{(FluxAvgRep}\SpecialCharTok{\textasciitilde{}}\NormalTok{Max.Ht, }\AttributeTok{family =} \StringTok{"poisson"}\NormalTok{, }\AttributeTok{data=}\NormalTok{sed\_SAPA)}
\NormalTok{sedvegglm}
\end{Highlighting}
\end{Shaded}

\begin{verbatim}
## 
## Call:  glm(formula = FluxAvgRep ~ Max.Ht, family = "poisson", data = sed_SAPA)
## 
## Coefficients:
## (Intercept)       Max.Ht  
##     3.09487     -0.01325  
## 
## Degrees of Freedom: 182 Total (i.e. Null);  181 Residual
## Null Deviance:       7605 
## Residual Deviance: 7307  AIC: 7914
\end{verbatim}

\begin{Shaded}
\begin{Highlighting}[]
\NormalTok{sedvegglm\_AIC\_text }\OtherTok{\textless{}{-}} \FunctionTok{paste}\NormalTok{(}\StringTok{"AIC: 7914"}\NormalTok{)}
\NormalTok{sedvegglm\_AIC\_text}
\end{Highlighting}
\end{Shaded}

\begin{verbatim}
## [1] "AIC: 7914"
\end{verbatim}

\begin{Shaded}
\begin{Highlighting}[]
\NormalTok{sedvegglm\_coefficient\_Max.Ht }\OtherTok{\textless{}{-}} \FunctionTok{paste}\NormalTok{(}\StringTok{"Coefficient for Max Height: {-}0.01325 "}\NormalTok{)}
\NormalTok{sedvegglm\_coefficient\_Max.Ht}
\end{Highlighting}
\end{Shaded}

\begin{verbatim}
## [1] "Coefficient for Max Height: -0.01325 "
\end{verbatim}

\begin{Shaded}
\begin{Highlighting}[]
\FunctionTok{plot}\NormalTok{(sedvegglm)}
\end{Highlighting}
\end{Shaded}

\includegraphics{Independent-Project-Final-Project-Savannah-Miller_files/figure-latex/Plot model diagnostics-1.pdf}
\includegraphics{Independent-Project-Final-Project-Savannah-Miller_files/figure-latex/Plot model diagnostics-2.pdf}
\includegraphics{Independent-Project-Final-Project-Savannah-Miller_files/figure-latex/Plot model diagnostics-3.pdf}
\includegraphics{Independent-Project-Final-Project-Savannah-Miller_files/figure-latex/Plot model diagnostics-4.pdf}

\begin{Shaded}
\begin{Highlighting}[]
\FunctionTok{library}\NormalTok{(glmmTMB)}
\end{Highlighting}
\end{Shaded}

\begin{verbatim}
## Warning: package 'glmmTMB' was built under R version 4.4.3
\end{verbatim}

\begin{Shaded}
\begin{Highlighting}[]
\NormalTok{sed\_glmm }\OtherTok{=} \FunctionTok{glmmTMB}\NormalTok{(FluxAvgRep}\SpecialCharTok{\textasciitilde{}}\NormalTok{Max.Ht }\SpecialCharTok{+}\NormalTok{ Distance }\SpecialCharTok{+}\NormalTok{ (}\DecValTok{1}\SpecialCharTok{|}\NormalTok{Site) }\SpecialCharTok{+}\NormalTok{ (}\DecValTok{1}\SpecialCharTok{|}\NormalTok{Season) }\SpecialCharTok{+}\NormalTok{ (}\DecValTok{1}\SpecialCharTok{|}\NormalTok{Year), }\AttributeTok{family =} \FunctionTok{poisson}\NormalTok{(), }\AttributeTok{data=}\NormalTok{sed\_SAPA)}
\FunctionTok{summary}\NormalTok{(sed\_glmm)}
\end{Highlighting}
\end{Shaded}

\begin{verbatim}
##  Family: poisson  ( log )
## Formula:          FluxAvgRep ~ Max.Ht + Distance + (1 | Site) + (1 | Season) +  
##     (1 | Year)
## Data: sed_SAPA
## 
##       AIC       BIC    logLik -2*log(L)  df.resid 
##    6436.4    6455.7   -3212.2    6424.4       177 
## 
## Random effects:
## 
## Conditional model:
##  Groups Name        Variance Std.Dev.
##  Site   (Intercept) 0.84222  0.9177  
##  Season (Intercept) 0.01034  0.1017  
##  Year   (Intercept) 0.07910  0.2812  
## Number of obs: 183, groups:  Site, 2; Season, 2; Year, 2
## 
## Conditional model:
##              Estimate Std. Error z value Pr(>|z|)    
## (Intercept)  2.537925   0.683466   3.713 0.000205 ***
## Max.Ht       0.013382   0.001091  12.271  < 2e-16 ***
## Distance    -0.026909   0.001343 -20.034  < 2e-16 ***
## ---
## Signif. codes:  0 '***' 0.001 '**' 0.01 '*' 0.05 '.' 0.1 ' ' 1
\end{verbatim}

\begin{Shaded}
\begin{Highlighting}[]
\NormalTok{sed\_glmm2 }\OtherTok{=} \FunctionTok{glmmTMB}\NormalTok{(FluxAvgRep}\SpecialCharTok{\textasciitilde{}}\NormalTok{Max.Ht }\SpecialCharTok{+}\NormalTok{ Distance }\SpecialCharTok{+}\NormalTok{ (}\DecValTok{1}\SpecialCharTok{|}\NormalTok{Site) }\SpecialCharTok{+}\NormalTok{ (}\DecValTok{1}\SpecialCharTok{|}\NormalTok{Season), }\AttributeTok{family =} \FunctionTok{poisson}\NormalTok{(), }\AttributeTok{data=}\NormalTok{sed\_SAPA)}
\FunctionTok{summary}\NormalTok{(sed\_glmm2)}
\end{Highlighting}
\end{Shaded}

\begin{verbatim}
##  Family: poisson  ( log )
## Formula:          FluxAvgRep ~ Max.Ht + Distance + (1 | Site) + (1 | Season)
## Data: sed_SAPA
## 
##       AIC       BIC    logLik -2*log(L)  df.resid 
##    6499.6    6515.7   -3244.8    6489.6       178 
## 
## Random effects:
## 
## Conditional model:
##  Groups Name        Variance Std.Dev.
##  Site   (Intercept) 0.81632  0.9035  
##  Season (Intercept) 0.04726  0.2174  
## Number of obs: 183, groups:  Site, 2; Season, 2
## 
## Conditional model:
##              Estimate Std. Error z value Pr(>|z|)    
## (Intercept)  2.663151   0.657876   4.048 5.16e-05 ***
## Max.Ht       0.013229   0.001090  12.135  < 2e-16 ***
## Distance    -0.026810   0.001339 -20.023  < 2e-16 ***
## ---
## Signif. codes:  0 '***' 0.001 '**' 0.01 '*' 0.05 '.' 0.1 ' ' 1
\end{verbatim}

\begin{Shaded}
\begin{Highlighting}[]
\NormalTok{sed\_glmm3 }\OtherTok{=} \FunctionTok{glmmTMB}\NormalTok{(FluxAvgRep}\SpecialCharTok{\textasciitilde{}}\NormalTok{Max.Ht }\SpecialCharTok{+}\NormalTok{ Distance }\SpecialCharTok{+}\NormalTok{ (}\DecValTok{1}\SpecialCharTok{|}\NormalTok{Site), }\AttributeTok{family =} \FunctionTok{poisson}\NormalTok{(), }\AttributeTok{data=}\NormalTok{sed\_SAPA)}
\FunctionTok{summary}\NormalTok{(sed\_glmm3)}
\end{Highlighting}
\end{Shaded}

\begin{verbatim}
##  Family: poisson  ( log )
## Formula:          FluxAvgRep ~ Max.Ht + Distance + (1 | Site)
## Data: sed_SAPA
## 
##       AIC       BIC    logLik -2*log(L)  df.resid 
##    6560.7    6573.5   -3276.3    6552.7       179 
## 
## Random effects:
## 
## Conditional model:
##  Groups Name        Variance Std.Dev.
##  Site   (Intercept) 0.7714   0.8783  
## Number of obs: 183, groups:  Site, 2
## 
## Conditional model:
##              Estimate Std. Error z value Pr(>|z|)    
## (Intercept)  2.640829   0.621875   4.247 2.17e-05 ***
## Max.Ht       0.012083   0.001073  11.265  < 2e-16 ***
## Distance    -0.025855   0.001314 -19.680  < 2e-16 ***
## ---
## Signif. codes:  0 '***' 0.001 '**' 0.01 '*' 0.05 '.' 0.1 ' ' 1
\end{verbatim}

\begin{Shaded}
\begin{Highlighting}[]
\NormalTok{sed\_glmm4 }\OtherTok{=} \FunctionTok{glmmTMB}\NormalTok{(FluxAvgRep}\SpecialCharTok{\textasciitilde{}}\NormalTok{Max.Ht }\SpecialCharTok{+}\NormalTok{ Distance }\SpecialCharTok{+}\NormalTok{ (}\DecValTok{1}\SpecialCharTok{|}\NormalTok{Site) }\SpecialCharTok{+}\NormalTok{ (}\DecValTok{1}\SpecialCharTok{|}\NormalTok{Year), }\AttributeTok{family =} \FunctionTok{poisson}\NormalTok{(), }\AttributeTok{data=}\NormalTok{sed\_SAPA)}
\FunctionTok{summary}\NormalTok{(sed\_glmm4)}
\end{Highlighting}
\end{Shaded}

\begin{verbatim}
##  Family: poisson  ( log )
## Formula:          FluxAvgRep ~ Max.Ht + Distance + (1 | Site) + (1 | Year)
## Data: sed_SAPA
## 
##       AIC       BIC    logLik -2*log(L)  df.resid 
##    6443.6    6459.6   -3216.8    6433.6       178 
## 
## Random effects:
## 
## Conditional model:
##  Groups Name        Variance Std.Dev.
##  Site   (Intercept) 0.8415   0.9173  
##  Year   (Intercept) 0.1051   0.3242  
## Number of obs: 183, groups:  Site, 2; Year, 2
## 
## Conditional model:
##              Estimate Std. Error z value Pr(>|z|)    
## (Intercept)  2.509216   0.688852   3.643  0.00027 ***
## Max.Ht       0.012987   0.001080  12.024  < 2e-16 ***
## Distance    -0.026574   0.001333 -19.942  < 2e-16 ***
## ---
## Signif. codes:  0 '***' 0.001 '**' 0.01 '*' 0.05 '.' 0.1 ' ' 1
\end{verbatim}

\begin{Shaded}
\begin{Highlighting}[]
\NormalTok{sed\_glmm5 }\OtherTok{=} \FunctionTok{glmmTMB}\NormalTok{(FluxAvgRep}\SpecialCharTok{\textasciitilde{}}\NormalTok{Max.Ht }\SpecialCharTok{+}\NormalTok{ Distance }\SpecialCharTok{+}\NormalTok{ (}\DecValTok{1}\SpecialCharTok{|}\NormalTok{Year) }\SpecialCharTok{+}\NormalTok{ (}\DecValTok{1}\SpecialCharTok{|}\NormalTok{Season), }\AttributeTok{family =} \FunctionTok{poisson}\NormalTok{(), }\AttributeTok{data=}\NormalTok{sed\_SAPA)}
\FunctionTok{summary}\NormalTok{(sed\_glmm5)}
\end{Highlighting}
\end{Shaded}

\begin{verbatim}
##  Family: poisson  ( log )
## Formula:          FluxAvgRep ~ Max.Ht + Distance + (1 | Year) + (1 | Season)
## Data: sed_SAPA
## 
##       AIC       BIC    logLik -2*log(L)  df.resid 
##      7690      7706     -3840      7680       178 
## 
## Random effects:
## 
## Conditional model:
##  Groups Name        Variance Std.Dev.
##  Year   (Intercept) 0.039110 0.19776 
##  Season (Intercept) 0.001758 0.04193 
## Number of obs: 183, groups:  Year, 2; Season, 2
## 
## Conditional model:
##               Estimate Std. Error z value Pr(>|z|)    
## (Intercept)  3.0547298  0.1462453   20.89  < 2e-16 ***
## Max.Ht      -0.0066284  0.0009193   -7.21 5.58e-13 ***
## Distance    -0.0100252  0.0009072  -11.05  < 2e-16 ***
## ---
## Signif. codes:  0 '***' 0.001 '**' 0.01 '*' 0.05 '.' 0.1 ' ' 1
\end{verbatim}

\begin{Shaded}
\begin{Highlighting}[]
\FunctionTok{AIC}\NormalTok{(sed\_glmm)}
\end{Highlighting}
\end{Shaded}

\begin{verbatim}
## [1] 6436.426
\end{verbatim}

\begin{Shaded}
\begin{Highlighting}[]
\FunctionTok{AIC}\NormalTok{(sed\_glmm, sed\_glmm2, sed\_glmm3, sed\_glmm4, sed\_glmm5)}
\end{Highlighting}
\end{Shaded}

\begin{verbatim}
##           df      AIC
## sed_glmm   6 6436.426
## sed_glmm2  5 6499.628
## sed_glmm3  4 6560.655
## sed_glmm4  5 6443.579
## sed_glmm5  5 7689.995
\end{verbatim}

\begin{Shaded}
\begin{Highlighting}[]
\NormalTok{sed\_MxHt\_season }\OtherTok{\textless{}{-}} \FunctionTok{ggplot}\NormalTok{(sed\_SAPA, }\FunctionTok{aes}\NormalTok{(}\AttributeTok{x =}\NormalTok{ Max.Ht, }\AttributeTok{y =}\NormalTok{ FluxAvgRep, }\AttributeTok{color =}\NormalTok{ Season)) }\SpecialCharTok{+}
  \FunctionTok{geom\_point}\NormalTok{(}\AttributeTok{position =} \FunctionTok{position\_jitter}\NormalTok{(}\AttributeTok{w =} \FloatTok{0.05}\NormalTok{, }\AttributeTok{h =} \DecValTok{0}\NormalTok{)) }\SpecialCharTok{+}
  \FunctionTok{geom\_smooth}\NormalTok{(}\AttributeTok{method =} \StringTok{"lm"}\NormalTok{, }\AttributeTok{se =} \ConstantTok{FALSE}\NormalTok{) }\SpecialCharTok{+}
  \FunctionTok{scale\_y\_log10}\NormalTok{() }\SpecialCharTok{+}
  \FunctionTok{labs}\NormalTok{(}
    \AttributeTok{x =} \FunctionTok{expression}\NormalTok{(}\StringTok{"Maximum Height of Pickleweed "} \SpecialCharTok{*} \FunctionTok{italic}\NormalTok{(}\StringTok{"(Salicornia pacifica)"}\NormalTok{)),}
    \AttributeTok{y =} \StringTok{"Log Flux of Sediment Deposition"}\NormalTok{,}
    \AttributeTok{title =} \StringTok{"Sediment Deposition Rates Across Differences in Maximum Plant Height}\SpecialCharTok{\textbackslash{}n}\StringTok{ Between Two Seasons"}
\NormalTok{  )}

\NormalTok{sed\_MxHt\_season}
\end{Highlighting}
\end{Shaded}

\begin{verbatim}
## Warning in scale_y_log10(): log-10 transformation introduced infinite values.
## log-10 transformation introduced infinite values.
\end{verbatim}

\begin{verbatim}
## `geom_smooth()` using formula = 'y ~ x'
\end{verbatim}

\begin{verbatim}
## Warning: Removed 16 rows containing non-finite outside the scale range
## (`stat_smooth()`).
\end{verbatim}

\includegraphics{Independent-Project-Final-Project-Savannah-Miller_files/figure-latex/graphing the relationships-1.pdf}

\begin{Shaded}
\begin{Highlighting}[]
\NormalTok{sed\_DistPlant }\OtherTok{\textless{}{-}} \FunctionTok{ggplot}\NormalTok{(veg, }\FunctionTok{aes}\NormalTok{(}\AttributeTok{x =}\NormalTok{ Distance, }\AttributeTok{y =}\NormalTok{ Max.Ht, }\AttributeTok{color =}\NormalTok{ Site)) }\SpecialCharTok{+} 
  \FunctionTok{geom\_point}\NormalTok{(}\AttributeTok{position =} \FunctionTok{position\_jitter}\NormalTok{(}\AttributeTok{w =} \FloatTok{0.05}\NormalTok{, }\AttributeTok{h =} \DecValTok{0}\NormalTok{)) }\SpecialCharTok{+}
  \FunctionTok{scale\_y\_log10}\NormalTok{() }\SpecialCharTok{+} 
  \FunctionTok{labs}\NormalTok{(}
    \AttributeTok{x =} \StringTok{"Distance from Marsh Edge"}\NormalTok{,}
    \AttributeTok{y =} \FunctionTok{expression}\NormalTok{(}\StringTok{"Maximum Height of Pickleweed "} \SpecialCharTok{*} \FunctionTok{italic}\NormalTok{(}\StringTok{"(Salicornia pacifica)"}\NormalTok{)),}
    \AttributeTok{title =} \StringTok{"Sediment Deposition Rates Across Plant Structure}\SpecialCharTok{\textbackslash{}n}\StringTok{(Maximum Height)"}
\NormalTok{  )}

\NormalTok{sed\_DistPlant}
\end{Highlighting}
\end{Shaded}

\includegraphics{Independent-Project-Final-Project-Savannah-Miller_files/figure-latex/graphing the relationships-2.pdf}

\subsection{Summary of
Results/Conclusions}\label{summary-of-resultsconclusions}

\begin{Shaded}
\begin{Highlighting}[]
\NormalTok{sed\_MxHt\_site }\OtherTok{\textless{}{-}} \FunctionTok{ggplot}\NormalTok{(sed\_SAPA, }\FunctionTok{aes}\NormalTok{(}\AttributeTok{x =}\NormalTok{ Max.Ht, }\AttributeTok{y =}\NormalTok{ FluxAvgRep, }\AttributeTok{color =}\NormalTok{ Site)) }\SpecialCharTok{+}
  \FunctionTok{geom\_point}\NormalTok{(}\AttributeTok{position =} \FunctionTok{position\_jitter}\NormalTok{(}\AttributeTok{w =} \FloatTok{0.05}\NormalTok{, }\AttributeTok{h =} \DecValTok{0}\NormalTok{)) }\SpecialCharTok{+}
  \FunctionTok{geom\_smooth}\NormalTok{(}\AttributeTok{method =} \StringTok{"lm"}\NormalTok{, }\AttributeTok{se =} \ConstantTok{FALSE}\NormalTok{) }\SpecialCharTok{+}
  \FunctionTok{scale\_y\_log10}\NormalTok{() }\SpecialCharTok{+}
  \FunctionTok{labs}\NormalTok{(}
    \AttributeTok{x =} \FunctionTok{expression}\NormalTok{(}\StringTok{"Maximum Height of Pickleweed "} \SpecialCharTok{*} \FunctionTok{italic}\NormalTok{(}\StringTok{"(Salicornia pacifica)"}\NormalTok{)),}
    \AttributeTok{y =} \StringTok{"Log Flux of Sediment Deposition"}\NormalTok{,}
    \AttributeTok{title =} \StringTok{"Sediment Deposition Rates Across Differences in Maximum Plant Height}\SpecialCharTok{\textbackslash{}n}\StringTok{ Between Two Marshes"}
\NormalTok{  )}
\NormalTok{sed\_MxHt\_site}
\end{Highlighting}
\end{Shaded}

\begin{verbatim}
## Warning in scale_y_log10(): log-10 transformation introduced infinite values.
## log-10 transformation introduced infinite values.
\end{verbatim}

\begin{verbatim}
## `geom_smooth()` using formula = 'y ~ x'
\end{verbatim}

\begin{verbatim}
## Warning: Removed 16 rows containing non-finite outside the scale range
## (`stat_smooth()`).
\end{verbatim}

\includegraphics{Independent-Project-Final-Project-Savannah-Miller_files/figure-latex/final figure-1.pdf}

Within this graph, sediment deposition is marked with a log function as
there were some outliers within the data set, specifically at the San
Pablo site that made it difficult to observe patterns in the data.

The graph above and the GLMM outputs demonstrates that maximum height of
Pickleweed does play a role in sediment deposition, though site specific
differences play a role in influencing those relationships. Season and
Year also did play a role in the strength of these predictor variables,
however since some of the interest of this study is based off of
replication across site, this was the final graph that I chose.

Based off of the conclusions of the linear regressions, as well as the
comparisons of the outputs for the general linear mixed models (GLMM),
it is apparent that site, season, and year are strong influences on both
of the independent variables I measured, which includes maximum height
of an ecologically important marsh species, Pickleweed (\emph{Salicornia
pacifica}) and the average sediment deposition rates across a marsh bay
ward edge. This is demonstrated from the six GLMMs performed, where the
AIC values fell all within a close range, but only the model with all
three random effects of Site, Season, and Year had the lowest value and
was therefore the best fit model.

Based on the repeated outputs of coefficients for both distance and
maximum height of Pickleweed, it is observed that both of these
predictors are significant in terms of their effect on sediment
deposition, but in very different ways.

There appears to be a negative relationship with distance and sediment
deposition-meaning that the farther you go away from the marsh edge, the
less sediment is deposited on the marsh platform. This makes sense from
the ecological understanding, as the marsh edge is the commonly
inundated with water than other points in the marsh, and the edge is
more likely to encounter dynamic wave energy and scarps that suspended
sediment in the water column would first interact with prior to settling
out.

There also appears to be a positive relationship with maximum height and
sediment deposition. This also makes sense ecologically, as a taller
plant would create more of an obstruction to suspended sediment in the
water column, causing for it to settle out quicker than with a shorter
plant.

Both distance and maximum height are interrelated to other variables of
the marsh, such as slope and average tide levels throughout the marsh.
As demonstrated from the GLMM outputs, site specific characteristics
such as slope and tide level cannot be ignored as predictors for
sediment deposition rates, but they were not measured within this study.
Slope is not accounted for in this model, and future analyses could take
into account this additional variable as it could characterize the
landscape differences between marshes that would result in such
differences in both plant growth and sediment deposition. Slope is
related to the mean low low water tide levels of the marsh, which are
also pertinent variables to consider when measuring differences in
sediment deposition as well as plant growth differences.

These results demonstrate that I can reject my null hypothesis based on
the significance of the effect of distance and maximum height on
differences in sediment deposition rates!

I am excited to see in my thesis studies how many of these additional
variables I can measure to see if I can uncover at least one clear
pattern with my own data-that would be incredible. Thanks for this class
Jenna! I will sincerely miss it!

\end{document}
